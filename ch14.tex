\section{Chapter 14. Chemical kinetics}


\secttoc

{\footnotesize
\begin{multicols}{3}
\begin{compactenum}
    \item Factors affecting rate of reactions
    \item Rate of reaction given time and concentration
    \item Relate rates of product formation and reactant disappearance given
        balanced chemical equation
    \item Explain form and meaning of a rate law (reaction order, rate constant)
    \item Determine these given measure rates for various concentrations
    \item Integrate rate laws
    \item Relationship between rate constant of 1st order and half-life
    \item Activation energy affects a rate, Arrhenius equation
    \item Predict rate law for a multi-step mechanism given each step.
    \item Explain principles of catalysis
\end{compactenum}
\end{multicols}
}

\begin{mdframed}\subsection{Reaction Rates}
\begin{multicols}{2}
\begin{compactdesc}
    \item[Chemical kinetics] concerned with rates of reaction
    \item[Reaction rates] affected by physical states
        the more readily collisions occur, the more rapidly goes the reaction.
        Affected by reaction concentrations and temperature. Catalysts increase
        reaction rates by affecting activation rates.
    \item[homogeneous] same phase
    \item[heterogeneous] different phases
\end{compactdesc}
\end{multicols}
\end{mdframed}


\begin{mdframed}\subsection{Rates and Concentration}
\begin{multicols}{2}
    Rate is typically measured in molarity per second, change in
    concentration, ($M/s$).
\begin{compactdesc}
    \item[Average rate] of disappearance of A $=
        \frac{\Delta [\text{A}]}{\Delta t}$
    \item[Instantaneous rate] concentrations can be measured using \textbf{spectroscopy}.
    \item[Beer's law] concentration, c is directly proportional to absorbance, A.
        $A = \epsilon b c$, b is path length.
    \item[Stoichiometry] In reaction \ce{A -> B} the rate of appearance of product
        is the rate of disappearance of reactant.
    \item[Relative rates] In reaction \ce{aA + bB -> cC + dD}, the rates are
        $   -\frac{1}{a}\frac{\Delta A}{\Delta t}
          = -\frac{1}{b}\frac{\Delta B}{\Delta t}
          =  \frac{1}{c}\frac{\Delta C}{\Delta t}
          =  \frac{1}{d}\frac{\Delta D}{\Delta t}
        $. This does not hold true if intermediate substances are formed in
        significant amounts.
    \item[Rates are positive] by convention

    \item[(Differential) Rate Law] Typically of form $R = -k[A]^m[B]^n$
    \item[Rate constant] $k :: M^{-m - n + 1}s^{-1}$. Different for every reaction
        and temperature; $>10^9$ means fast, $<10$ means slow.
    \item[Reaction orders]
        $m, n$ usually 0, 1, 2. Can be negative or fractional.
    \item[Overall reaction order] $n + m$
    \item[Rate law from Initial rates]
        \(
            n = \frac {\ln \frac {Rate_1} {Rate_2} } {\ln \frac {[A]_1} {[A]_2} }
        \)
\end{compactdesc}
\end{multicols}
\end{mdframed}



\begin{mdframed}
\subsection{Concentration and Time}
\begin{multicols}{2}
    Integrated, linear rate laws for the first order reaction \ce{A -> \dots}
\begin{compactdesc}
    \item[First-order reaction]
        $\ln [A]_t = \ln [A]_0 - kt$.
    \item[Second order reaction]
        $\frac{1}{[A]_t} = \frac {1} {[A]_0} + kt$.
    \item[Zero-order reaction] $[A]_t = [A]_0 - kt$
    \item[Half life] of a first-order depends only on $k$. $t_{1/2} = -\frac{\ln 1/2} {k}$
\end{compactdesc}
\end{multicols}
\end{mdframed}





\begin{mdframed}\subsection{The Effect of Temperature on Rates}
\begin{multicols}{2}
\begin{compactdesc}
    \item[Collision model] based on kinetic molecular theory. Molecules
        must collide to react, so goes the theory. More energy/temp means
        more collisions, faster rate. Molecules must be aligned correctly on impact.
    \item[Activation energy, $E_a$] needs to be achieved.
        Must also be moving fast enough and in the correct orientation.
        Lower $E_a$ means faster reactions.
        Never negative.
    \item[Rate does not depend on] $\Delta H$, only on $E_a$
    \item[Activated complex/transition state] Enough energy in the right
        direction, now the molecules may react together.
        Highest potential energy. It's all downhill from here!
    \item[Arrhenius equation] Rate constant incorporates all of these:
        $k = Ae^{-E_a / RT}$. A is the \textbf{frequency factor}, constant as
        temperature is varied. Can be used to find different $k$s.
        $R = 8.314 J \cdot mol^{-1} \cdot K^{-1} $
        $= 0.082057L \cdot atm \cdot K^{-1}\cdot mol^{-1}$
    \item[Solve for $k$] at different temperatures. $A$ and $E_a$ remain
        constant:
        \[
            \ln \frac{k_1}{k_2} = \frac{E_a}{A} \big( 1/T_2 - 1/T_1 \big)
        \]
\end{compactdesc}
\end{multicols}
\end{mdframed}



\begin{mdframed}\subsection{Reaction Mechanisms}
\begin{multicols}{2}
\begin{compactdesc}
    \item[Reaction mechanism] Steps which constitute a reaction.
        Order in which bonds are modified. And change in relative positions.
    \item[Elementary reactions] Single step. Defined by \# of molecules
        colliding: \textbf{uni/bi/ter}molecular.
        $>3$ molecules colliding is improbable.
    \item[Intermediate]
        \ce{NO2 + CO -> NO + CO2} is actually
        \ce{NO2 + NO2 -> NO3 + NO} then \ce{NO3 + CO -> NO2 + CO2}
        Together: \ce{2NO2 + NO3 + CO -> NO2 + NO3 + NO + CO2}
        Thus \ce{NO3} is an intermediate, common in both sides of the addition.
        Multi-step reactions have one or more such substances.
    \item[Rate laws] of elementary reactions have an overall reaction order
        equal to the number of molecules involved. Termolecular, $n + m = 3$.
        A single chemical equation cannot tell us if a reaction is elementary!
    \item[Rate-determining step] is the slowest. If initial, likely that its
        rate law will govern that of the overall.
    \item[Slow initial step] the rate is usually equal to the rate of the
        initial step.
\end{compactdesc}

If the initial step is fast, the rate law may be derived by assuming
equilibrium and solving the $k_1$ and $k_{-1}$ to get rid of the intermediate
substance \ce{NOBr2}.

Example: \[\ce{2NO + Br2 -> 2NOBr}\] could be a termolecular reaction, or
the fast equilibrium step and a slow final step
\[\ce{NO + Br2 <=>T[k1][k-1] NOBr2}\]
\[\ce{NOBr2 + NO ->T[k2] 2NOBr}\]
$k_1, k_{-1}$ reach
equilibrium meaning rate of (forward = reverse) and the total rate constant
$k = k_2 \frac{k_1}{k_{-1}}$. Two steps, but only uni/bi molecular is more likely.
35 or more steps are sometimes derived!
\end{multicols}
\end{mdframed}



\begin{mdframed}\subsection{Catalysts}
\begin{multicols}{2}
\begin{compactdesc}
    \item[Catalyst] changes the speed of (initial) activation of a reaction without being changed itself
        Usually done by lowering $E_a$. Opposite = inhibitor.
    \item[Homogeneous catalyst] same phase as reactants. Not so for a
        \textbf{Heterogeneous catalyst} (first step usually adsorption of
        reactants)
    \item[Adsorption] (think adhesion), how fast a substance is bound to a
        surface.
    \item[Enzymes] marvelously efficient biological catalysts. Usually huge
        proteins, kilo to mega amu. Very selective. Faster than nonbiologicals,
        $10^3 - 10^7$ reactions per molecule per second (turnover number).
    \item[Substrates] substances reacting at the active site
    \item[Active site] location of catalysis.
    \item[Lock-and-key model] Explanation for specificity of an enzyme.
\end{compactdesc}
\end{multicols}
\end{mdframed}
