\section{Chapter 20. Electrochemistry}

%  In aqueous solutions, if the standard reduction potential of the metal is less than that of water, then the metal does not participate in the electrolysis, and water is reduced to hydrogen gas. If the standard oxidation potential of the anion is less than that of water, then water is oxidized to oxygen and the anion does not participate in the reaction.

\secttoc

{\footnotesize \begin{multicols}{3}\begin{compactenum}
\item ID oxidation, reduction, oxidizing agent, reducing agent
\item Complete, balance redox equations using half-reactions
\item Sketch voltaic cells, ID cathode, anode, direction of $e^-$ motion
\item Standard emfs from standard reduction potential
\item Reduction potentials to predict if redox is spontaneous
\item Relate $E^\circ_\text{cell}, \Delta G^\circ$ to equilibrium constants
\item Calculate emf under nonstandard conditions
\item ID components of common batteries
\item Construction, explanation of lithium-ion battery
\item Construction, explanation of fuel cell
\item Corrosion, prevent with cathode protection
\item Reactions in electrolytic cells
\item Amount of products, reactants in redox and electric charge
\end{compactenum}\end{multicols}}


\begin{mdframed}
\begin{multicols}{2}
\subsection{Oxidation States and Redox}
\begin{compactdesc}
    \item[electrochemistry] study of electricity and chemical reactions
    \item[OIL RIG] oxidized is loss of electrons, reduction is gaining
        electrons.
    \item[reducing agent] causes reduction, it is oxidized to accomplish this
    \item[oxidizing agent] causes oxidization, it is reduced to accomplish this
\end{compactdesc}
\end{multicols}
\end{mdframed}






\begin{mdframed}
\begin{multicols}{2}
\subsection{Balancing Redox}
\begin{compactdesc}
    \item[half-reactions] a redox can be split into a reducing and into an
        oxidizing equation:
        \[
            son of bitch
        \]
    \item[steps to balance in aqueous]
        \begin{compactenum}
        \item
        \end{compactenum}
    \item[balance in basic aqueous]
    \item[adding half-reactions] electrons should cancel to reveal the
        balanced equation
\end{compactdesc}
\end{multicols}
\end{mdframed}






\begin{mdframed}
\begin{multicols}{2}
\subsection{voltaic cells}
\begin{compactdesc}
    \item[]
    \item[]
    \item[]
    \item[]
    \item[]
\end{compactdesc}
\end{multicols}
\end{mdframed}






\begin{mdframed}
\begin{multicols}{2}

\subsection{Voltage under Standard Conditions}
\begin{compactdesc}
    \item[standard conditions] 298K, 1atm, 1M
    \item[]
    \item[]
    \item[]
    \item[]
\end{compactdesc}

\subsection{Free Energy and Redox Reactions}
\begin{compactdesc}
    \item[Free energy $\Delta G$]
        \[
            \Delta G = -nFE
        \]
    \item[Faraday constant] $F = 96,485 C/mol$
    \item[Positive E] spontaneous
    \item[Negative E] non-spontaneous
    \item[1 Watt (W)] 1 J/s
\end{compactdesc}


\subsection{Voltage under Nonstandard Conditions}
\begin{compactdesc}
    \item[emf varies] with temperature and concentrations
    \item[Nernst equation] Let n be the moles of $e^{-}$ exchanged and Q be
        calculated similar to equil. constant $k$ (use atm pressure without
        conversion):

        \[
            E = E^\circ - (RT/nF) \ln Q
        \]
    \item[At 298K]
        \[
            E = E^\circ - (0.0592/n) \ln Q
        \]
\end{compactdesc}
\end{multicols}
\end{mdframed}






\begin{mdframed}
\begin{multicols}{2}
\subsection{Batteries and Fuel Cells}
\begin{compactdesc}
    \item[battery] self contained electrochemical power source.
        Based on a variety of redox reactions.
    \item[Primary cells] cannot be recharged
    \item[Secondary cells] can be
    \item[Common primary cell]  alkaline dry cell
    \item[Common secondary cells] Lead-acid, \ce{Ni-Cd}, Nickel-metal hydride
        and Lithium-ion.
    \item[Fuel cells] voltaic cells that need to be continuously supplied with
        reactants (such as \ce{H2}) for a redox reaction.
\end{compactdesc}


\subsection{Corrosion}
\begin{compactdesc}
    \item[corrosion] undesirable redox reaction.
    \item[cathodic protection] protecting a metal by covering it using another
        that more readily undergoes oxidation.
    \item[example] galvanized steel is \ce{Fe} covered in \ce{Zn}, a
        sacrificial anode in the redox reaction.
\end{compactdesc}

\end{multicols}
\end{mdframed}






\begin{mdframed}
\begin{multicols}{2}
\subsection{Electrolysis}
\begin{compactdesc}
    \item[electrolysis reaction]
    \item[electrolytic cell]
    \item[current carrying medium] molten salt or electrolyte solution
    \item[predict products] by comparing potentials of the red. and oxi. processes.
    \item[active electrodes] are involved in the reaction
    \item[quantity of substance formed] related to total current. $1\frac{C}{s} = 1A$
\end{compactdesc}
\end{multicols}
\end{mdframed}
