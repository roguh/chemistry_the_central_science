\section{Chapter 4. Reactions in Aqueous Solution}

\secttoc

{\footnotesize
\begin{multicols}{3}
\begin{compactenum}
    \item ID compounds as acids or bases, strong weak or non-electrolyte?
    \item Recognize reaction types, simple acid-base, precipitation and
        redox (OILRIG) reactions
    \item Calculate molarity, use it to find volume and moles
    \item Carry out a dilution to achieve a desired solution
    \item Perform and interpret results of a titration
\end{compactenum}
\end{multicols}
}


\begin{mdframed}
\subsection{Solubility Guidelines}
\begin{multicols}{2}
\begin{tabular}{ll}
    soluble & exceptions                                    \\
    \hline
    \ce{NO3^-}                                     &        \\
    \ce{CH3COO^-}                                  &        \\
    alkali metals                                  &        \\
    \hline
    all following & unless with \ce{Hg2^{2+}}, \ce{Pb^{2+}} \\
    \ce{Cl^-}      & unless with \ce{Ag+}                   \\
    \ce{Br^-}      & unless with \ce{Ag+}                   \\
    \ce{I^-}       & unless with \ce{Ag+}                   \\
    \ce{SO4^{2-}}  & unless with \ce{Sr^{2+}}, \ce{Ba^{2+}} \\
\end{tabular}

\begin{tabular}{ll}
 insoluble  & exceptions \\
\hline
 all  following & unless with \ce{NH4+}, alkali metal \\
 \ce{CO3^{2-}}  &  \\
 \ce{PO4^{3-}}  &  \\

 all following & unless with \ce{Ca^{2+}}, \ce{Sr^{2+}}, \ce{Ba^{2+}}  \\
 \ce{S^{2-}} &  \\
 \ce{OH^-}   &  \\
\end{tabular}
\end{multicols}


\subsection{Strong Acids and Bases}
\begin{compactdesc}
\item[Strong acids] HI, HBr, HCl, \ce{HClO3}, \ce{HClO4}, \ce{HNO3}, \ce{H2SO4} (first proton only)
\item[Strong bases] group 1A metal hydroxides \ce{KOH}, group 2A heavy metal
    hydroxides (beginning at \ce{Ca(OH)2})
\item[Weak acids, bases] all else
\end{compactdesc}
\end{mdframed}






\begin{mdframed}
\subsection{Solution basics}
\begin{multicols}{2}
\begin{compactdesc}
\item[Solvent] more of this
\item[Solute] dissolved in solvent
\item[Molarity] moles per Litre
\item[Dilution] $M_\text{conc}V_\text{conc} = M_\text{dilute}V_\text{dilute}$
\item[Electrolyte] fancy for salt. Forms ions in water by dissociating.
\item[Weak electrolytes] oscillate, only a fraction of itself is dissociated.
\item[Precipitation] pairs of oppositely charged ions attract each other
    to form a solid (salt). Look for any insoluble products.
    A type \textbf{metathesis} reaction \ce{AX + BY -> AY + BX}
\item[Ionic equation] Split any dissociated (aq) molecules into ions.
    Cancel out ions present on both sides of equation, these are
    \textbf{spectator ions}.
\item[Example ionic equation] \ce{Pb(NO3)2(aq) + 2KI(aq) -> PbI2 (s) + 2KNO3(aq)}
    can be converted to \ce{Pb^{2+}(aq) + 2I^-(aq) -> PbI2(s)}
    with \ce{K+} and \ce{(NO3)2^{2+}} as spectator ions.
\end{compactdesc}
\end{multicols}
\end{mdframed}






\begin{mdframed}
\begin{multicols}{2}
\subsection{Acid-Base Reactions}
\begin{compactdesc}
\item[Acids] form \ce{H+} in water
\item[Bases] form \ce{OH^-} in water
\item[Acids and Bases] combined \ce{HA + BOH -> H2O + AB}
\item[Gas formation] is possible \ce{2HCl(aq) + Na2S(aq) -> H2S(g) + 2NaCl(aq)}
\end{compactdesc}

\subsection{Oxidation-Reduction Reactions}
\begin{compactdesc}
\item[Metal activity] increases to the top-left. Higher means easier to
    oxidize.
\item[OIL-RIG] oxidation is loss of $e^-$, reduction is gain of $e^-$
\item[Oxidation number] is an artificial (negated) $e^-$ count.
    \begin{compactdesc}
    \item[O] -2, except in \ce{O2^{2+}} where each O has -1.
    \item[H] +1 with metals, -1 with nonmetals
    \item[F] -1
    \item[other halogens] -1, can be positive with oxygen
    \item[elemental] 0
    \item[monatomic ion] charge
    \item[polyatomic ion] sum of oxidation numbers is charge
    \end{compactdesc}
\end{compactdesc}
\end{multicols}
\end{mdframed}





