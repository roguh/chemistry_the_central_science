\section{Chapter 1. Units and Significant Figures}

{\footnotesize
\begin{multicols}{3}
\begin{compactenum}
    \item Elements,  compounds,  hetero/homogeneous mixtures
    \item Common elements
    \item Common prefixes
    \item Significant figures OHYEA
    \item SI units and non-SI units common in chemistry
\end{compactenum}
\end{multicols}
}

\begin{mdframed}
\begin{multicols}{2}
\begin{compactdesc}
    \item[Common elements] C, F, H, N, I, O, P, S, Al, Br, Ca, Cl, He, Li, Mg, Si, Cu, Fe, Pb,
        Hg, K, Ag, Na, Sn
    \item[Common prefixes] Yocto (15),  Tera,  Giga,  Mega,  kilo,  deci,  centi,
        mill,  $\mu$icro,  nano,  \r{A}ngstrom(-10,  non-SI),  p,  femto (-15),  atto,
        zepto (-21)
    \item[SI units] kg, m, s or sec, K, mol, A, cd (candela)
    \item[Common in chem:] L, 1000 cubic centimeters
    \item[Sig.fig. addition] keep lowest decimal sig.fig.s. $1.0 + 221.31 = 222.3$
    \item[Sig.fig. multiplication] keep lowest sig.fig.s
    \item[Sig.fig. logarithms] sig.figs of mantissa expressed in scientific
        notation equals the number of significant figures to the right of the decimal
        $\log(2.73 ×10^5) = \log(2.73) + \log(10^5) = 0.436 - 5.00000...$.
\end{compactdesc}
\end{multicols}
\end{mdframed}
