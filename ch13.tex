\section{Chapter 13. Properties of Solutions}


\secttoc

{\footnotesize
\begin{multicols}{3}
\begin{compactenum}
    \item Enthalpy/entropy changes affect solution formation
    \item IMFs and solubility, like dissolves like
    \item Equilibrium in the solution process, solubility of a solute
    \item Temperature and solid, liquid or gas solubility
    \item Partial pressure of a gas and its solubility
    \item Molarity, molality, mole fraction, percent composition, ppm, inter-convert!
    \item Colligative property? How do non-electrolytes and electrolytes affect?
    \item Vapor pressure of a solvent over soln
    \item BP elevation, FP depression
    \item Osmotic pressure of a solution
    \item Solution vs colloid
    \item Similarities between motions of gas molecules and the motions of
        colloids in a liquid
\end{compactenum}
\end{multicols}
}

\begin{mdframed}\subsection{Solution Process}
\begin{multicols}{2}\begin{compactdesc}
    \item[Natural tendency] formation of solutions is favored by the increase in
        entropy that accompanies mixing.
    \item[Intermolecular interactions] solute-solute, solvent-solvent,
        solvent-solute. The first two must be defeated to disperse and make room.
    \item[Solvation] when ions/molecules of solute are split and surrounded by
        solvent.
    \item[Solvation] solute is surrounded by solvent molecules.
    \item[Hydration] solvation with water as solvent
    \item[Energy cost] Solute and solvent must break down, then they must be
        mixed.
        $ \Delta H_{soln}
        = \Delta H_{solute} + \Delta H_{solvent} + \Delta H_{mix}$

        Think:
            \ce{solute$_n$ <=> $n$ solute} takes $\Delta H_{solute}$
            \ce{solvent$_m$ <=> $m$ solvent} takes $\Delta H_{solvent}$
            \ce{$n$ solute + $m$ solvent <=> solution} takes $\Delta H_{mix}$
    \item[Exothermic] $\Delta H_{mix} < 0$
    \item[Endothermic] $\Delta H_{solvent/solute} > 0$ need energy.
    \item[Spontaneity] If overall exothermic, solvation occurs spontaneously.
        Explains why \textbf{likes dissolve likes}. Similar IMFs have similar
        costs.
\end{compactdesc}\end{multicols}\end{mdframed}


\begin{mdframed}\subsection{Saturated solutions and solubility}
\begin{multicols}{2}\begin{compactdesc}
    \item[Equilibrium] \ce{solute + solvent <=> solution}
    \item[Crystallization] when solute is reverted to original state.
        ``Undissolved.'' Happens at the same time as solvation. Visible when
        solvent is spent and cannot maintain solvation interactions with enough
        solute.
    \item[Saturation] Equilibrium with undissolved solute. Additional solute
        will not dissolve (unless you add heat!). \textbf{Make a super-saturated: }
        add heat, lower heat, creates unstable solution.
    \item[Solubility] how much solute an amount of solvent can dissolve.
\end{compactdesc}\end{multicols}\end{mdframed}


\begin{mdframed}\subsection{Factors affecting solubility}
\begin{multicols}{2}\begin{compactdesc}
    \item[Like dissolves like] the stronger the interactions between solvent
        and solute, the greater the solubility. Polar dissolves polar; non-polar,
        non-polar.
    \item[Miscible] liquids that mix in all proportions. May be
        \textbf{immiscible.}
    \item[Pressure] liquid and solid solubilities are unaffected. Gas solubility
        is related to its partial pressure above liquid.
    \item[Henry's law] molarity of gas $=$ constant $\cdot$ partial pressure
        $S_g = k P_g$. Constant $k$ depends on temperature, solute and solvent.
    \item[Temperature] increase  increases solubility of most \textbf{solids}
        in water. There are exceptions \ce{Ce2 (S O4)3}
    \item[Temperature] increase decreases solubility of most \textbf{gases}
        in water
\end{compactdesc}\end{multicols}\end{mdframed}


\begin{mdframed}\subsection{Expressing solution concentrations}
\begin{multicols}{2}\begin{compactdesc}
    \item[Mass percentage]
        $\frac {\text{mass of component in soln}}
               {\text{total soln mass}} \cdot 100\% $
    \item[Parts per million/billion] $ \frac{mg}{kg} $, also
        $\frac {\text{mass of component in soln}}
               {\text{total soln mass}} \cdot 10^6 $
    \item[Mole fraction]
        $\frac {\text{moles of component in soln}}
               {\text{total moles}} $
    \item[Molality] Commonly used, independent of temperature!
        $\frac {\text{moles of solute}}
               {\text{kg of solvent}} $, little m.
    \item[Molarity]
        $\frac {\text{moles of solute}}
               {\text{litres of soln}} $, big M.
\end{compactdesc}\end{multicols}\end{mdframed}


\begin{mdframed}\subsection{Colligative Properties}
\begin{multicols}{2}\begin{compactdesc}
    \item[Colligative property] depend on concentration of solute, type of
        solvent. Not on the type of solute.
    \item[Raoult's law, vapor pressure depression]
        Applies to \emph{nonvolatile solute}
        \[
            P_\text{solution} = X_\text{solvent} P_\text{solvent}
        \]
        \[
            \Delta P = X_\text{solute} P_\text{solvent}
        \]
        \[
            X_\text{y} = \frac{\text{moles of y}}{\text{total moles}}
        \]
    \item[Solutions w/ two or more volatile] components, the vapor pressure of
        the solution is the sum of the vapor pressure of each component as
        calculated by Raoult's law.
    \item[Ideal solution] obey Raoult's law. Solute-solute, solvent-solvent and
        solute-solvent interactions all equal.
    \item[Molal boiling point elevation constant] BP of a solution is
        higher than that of the solvent. $m$ is molality.
        \[
            \Delta T_b = T_{b(\text{solution})} - T_{b(\text{solvent})}
            = i K_b m
        \]
    \item[Molal freezing point depression constant] FP of a solution is lower
        than that of the solvent. $m$ is molality.
        \[
            \Delta T_f = T_{f(\text{solution})} - T_{f(\text{solvent})}
            = - i K_f m
        \]
    \item[van't Hoff Factor] number of fragments a solute breaks up into for
        that particular solvent. Usually anion/cation dissociated.
    \item[True van't Hoff Factor] $i = \frac
        {\Delta T_f \text{measured}}
        {\Delta T_f \text{theoretical}}$
    \item[Osmosis] Solvent molecules pass through semipermeable membrane
        between two solutions of differing concentration. Solvent always
        goes to solution with lower solvent concentration. Want equilibrium
        (isotonic). Hypotonic: lower osmotic pressure, hypertonic: more
        concentrated.
    \item[Osmotic pressure] $\Pi = i \big( \frac{n}{V_\text{soln}} \big) RT
                                 = iM_\text{molarity}RT$
\end{compactdesc}\end{multicols}\end{mdframed}


\begin{mdframed}\subsection{Colloids}
\begin{multicols}{2}

    Some mixtures appear to initially dissolve, but gravity separates solute
    from solvent.
    \begin{compactdesc}
    \item[Colloids] between homogeneous mixtures and true solutions. Large
        solvent molecules/particles.
    \item[Tyndall effect] Large enough to \textbf{scatter light}, Tyndall
        effect.
    \item[Hydrophilic] folds to keep hydrophobic groups away from water.
        Polar groups go to surface. Usually have \ce{N} or \ce{O} and a charge.
    \item[Hydrophobic] can only be dispersed if stabilized. Otherwise, they
        run away. One method is via adsorption of ions to the surface. This
        repels particles from each other and causes interactions with the water.
    \item[Emulsion] a suspension of one liquid in another, like Milk.
    \item[Adsorption] think adhesion. Sticking stuff to surface.
    \item[Brownian motion] collisions cause colloid particles to exhibit random
        motion
\end{compactdesc}\end{multicols}\end{mdframed}



