\section{Chapter 16. Acid-Base Equilibria}


\secttoc

{\footnotesize
\begin{multicols}{3}
\begin{compactenum}
    \item ID Arrhenius acids and bases
    \item Describe nature of hydrated proton, either \ce{H+ (aq)} or \ce{H3O+ (aq)}
    \item ID Br{\o}nsted-Lowry acids and bases and ID conjugate acid-base pairs
    \item Correlate the strength of an acid to the strength of its conjugate
        base
    \item Equilibrium position of a proton-transfer relates to the strengths of
        acids and bases involved
    \item Auto-ionization of water and explain how [\ce{H3O+}] and [\ce{OH-}] are
        related by $K_W$
    \item Calculate the pH given [\ce{H3O+}] and [\ce{OH-}]
    \item Calculate the pH of a strong acid or base given its concentration
    \item Relate $K_a$ and/or $K_b$ for a weak acid/base and its concentration
        and the pH
    \item Calculate $K_b$ for a weak base given $K_a$ of its conjugate acid and
        vice versa
    \item Predict whether an aqueous solution of a salt will be acidic, basic
        or neutral
    \item Predict the relative strength of a series of acids from their
        molecular structures
    \item ID Lewis acids and bases

\end{compactenum}
\end{multicols}
}


\begin{mdframed}
\begin{multicols}{2}
\subsection{Arrhenius Acids and Bases}
\begin{compactdesc}
    \item[Acid] increases concentration of \ce{H+} ions.
    \item[Bases] increase the concentration of \ce{OH-} ions
    \item[Examples] \ce{HCl (g) -> H+ (aq) + Cl- (aq)}
                 \\ \ce{NaOH -> Na+ (aq) + OH- (aq)}
\end{compactdesc}

\subsection{Br{\o}nsted-Lowry Acids and Bases}
\begin{compactdesc}
    \item[Acid-base reactions] involve the transfer of protons from one
        substance to another
    \item[Acid] donates a proton to another substance
    \item[Base] receives a proton
    \item[Hydronium ion] closer to the reality of aqueous solutions.
    \item[Used interchangeably] \ce{H+ (aq)} and \ce{H3O+ (aq)}
    \item[Water acts like] a Br{\o}nsted-Lowry base when a Br{\o}nsted-Lowry
        acid is dissolved:
        \ce{HA + H2O (l) -> H3O+ (aq) + A- (aq)}.
    \item[Water also acts like] a Br{\o}nsted-Lowry acid when a
        Br{\o}nsted-Lowry base is dissolve:
        \ce{B + H2O (l) -> BH+ (aq) + OH- (aq)}
    \item[Amphoteric] substances can behave as either acids and bases.
        Basic if in the presence of something more acidic, acidic if the other
        is more basic.
    \item[Amphiprotic] Either a proton acceptor or donor. Although an
        amphiprotic species must be amphoteric, the
        converse is not true.
    \item[Amphiprotic] \ce{HCO3^- , HS^-, HPO4^2-, HF, H2O}
    \item[Conjugate acid] every base converts into one. \ce{H3O+} is the c.a.
        of \ce{H2O}
    \item[Conjugate base] every acid converts into one. \ce{OH-} is the c.b.
        of \ce{H2O}
    \item[Relative Strengths of Acids and Bases]
        The stronger the acid/base, the weaker its conjugate base/acid.
        \begin{compactenum}
        \item Strong acid's conjugate base shows negligible basicity
        \item Weak acid's conjugate base is a weak base
        \item Negligibly acidic substance's conjugate is a strong base
        \end{compactenum}
    \item[Leveling effect] Stronger acids react with water to produce
        \ce{H3O+} and stronger bases react to produce \ce{OH-}. These
        are the strongest acids and bases that can exist in water.
    \item[Equilibrium favors] transfer of protons to form weaker acids and
        bases.
\end{compactdesc}

\end{multicols}
\end{mdframed}



\begin{mdframed}
\begin{multicols}{2}
\subsection{Auto-ionization of Water}
\begin{compactdesc}
    \item[Autoionization] \ce{H2O (l) + H2O (l) <=> OH- (aq) + H3O+ (aq)}
    \item[Ion-product constant] the equilibrium constant for water
        $k_w = 1\cdot10^{-14}$ at 25 degrees Celsius.
    \item[Neutral solution] \ce{[H+] = [OH^-]}
\end{compactdesc}

\subsection{pH and pOH Scale}
\begin{compactdesc}
\item[pH] is equal to \ce{-\log_{10}[H+]}
\item[pOH] is equal to \ce{-\log_{10}[OH^-]}
\item[pH + pOH] at 25 degrees Celsius is always 14.
\item[Negative pH] indicates a very strong acid
\item[pH greater than 14] indicates a very strong base
\item[pH may be measured] by detecting trace electric charge
\item[pH depends on] concentration of the acid or base
\end{compactdesc}

\end{multicols}
\end{mdframed}




\begin{mdframed}
\begin{multicols}{2}
\subsection{Strong acids and bases}
\begin{compactdesc}
\item[Most common strong acids] six mono-protic:
    (\ce{HClO3}, \ce{HClO4}, \ce{HCl}), \ce{HBr}, \ce{HI}, \ce{HNO3},
    and one diprotic: \ce{H2SO4} but only with the first proton
\item[Strong:] no equilibrium, balance lies entirely to the right
\item[Most common strong bases] ionic hydroxides of the alkali metals, and
    of the heavier alkaline earth metals (Calcium++). The latter are limited
    in solubility.
\end{compactdesc}
\end{multicols}
\end{mdframed}




\begin{mdframed}
\begin{multicols}{2}
\[k_a k_b = k_w\]

\subsection{Weak Acids and $k_a$}
\begin{compactdesc}
    \item[Equilibrium constant for acids] is called the acid-dissociation
        constant $k_a$. Larger means stronger. Need not be aqueous.
        Example for \ce{HA (aq) <=> H+ (aq) + A^- (aq)}:
        \[ k_a = \frac{\ce{[H+] [A^-]}}{\ce{[HA]}} \]

    \item[Percent ionization for acids] Larger also means stronger.
        \[\frac{\ce{[H+]_{equilibrium}}}{\ce{[HA]_{initial}}}\]
    \item[Polyprotic acids] can undergo more than one dissociation. \ce{H2SO3}
        has two protons to give away. It is always easier to remove the first
        proton.
    \item[p$K_a$] is equal to $-\log_{10} k_a$
\end{compactdesc}

\subsection{Weak Bases and $k_b$}
\begin{compactdesc}
     \item[Equilibrium constant for bases] is called the base-dissociation
        constant $k_b$. Larger means stronger. Only for water based!
        Example for \ce{B (aq) + H2O (l) <=> HB+ (aq) + OH^- (aq)}:
        \[ k_b = \frac{\ce{[BH+][OH^-]}}{\ce{[B]}} \]
    \item[Category One] neutral substances that have an atom with a non-bonding
        pair of electrons that can accept a proton. Most have Nitrogen.
        Includes Ammonia and Amines.
    \item[Amines] at least one \ce{N-H} bond in \ce{NH3} is replaced with
        \ce{N-C}.
    \item[Second category] anions of weak acids. The acid \ce{NaClO} has
        the conjugate base \ce{ClO-}.
    \item[p$K_b$] is equal to $-\log_{10} k_b$
\end{compactdesc}

\end{multicols}
\end{mdframed}




\begin{mdframed}
\begin{multicols}{2}
\subsection{Acid-Base Properties of Salt Solutions}
\begin{compactdesc}
\item[Nearly all salts are strong electrolytes] their acid-base properties
    are due to their cations and anions
\item[Hydrolysis] ions react with water to generate \ce{H+} or \ce{OH-}.
\item[An anion can be considered] the conjugate base of an acid \ce{A^-}.
    If it is not a strong acid, it is a weak acid.
\item[Polyatomic cations with one or more protons] can be considered the
    conjugate acids of weak bases.
\item[The larger the charge on the metal ion] the stronger the interaction
    between ion and oxygen of its hydrating water molecules. Facilitates
    proton transfer.
\item[Combined effect of cation and anion]:
    \begin{compactenum}
    \item Anion and cation don't react with water (both from strong a/b)?
        pH should be neutral. \ce{NaCl}, \ce{Ba(NO3)2}, \ce{RbClO4}
    \item Anion produces hydroxide ions, cation doesn't react (from weak acid, strong base)?
        pH should be basic. \ce{NaClO}, \ce{RbF}, \ce{BaSO3}
    \item Cation produces hydronium ions, anion doesn't react (from weak base, strong acid)?
        pH should be acidic. \ce{NH3NO3}, \ce{AlCl3}, \ce{Fe(NO3)3}
    \item Both anion and cation react in water (both from weak)?
        The pH of the solution depends on the relative abilities of the ions to
        react.
        \ce{NH4ClO}, \ce{Al(CH3COO)3}, \ce{CrF3}
    \end{compactenum}
\end{compactdesc}

\end{multicols}
\end{mdframed}


\begin{mdframed}
\begin{multicols}{2}

\subsection{Acid-Base Behavior and Chemical Structure}
\begin{compactdesc}
    \item[Strength of the \ce{H-A}] bond is the greatest indicator of
        acid strength
    \item[Acid Strength]
        \begin{compactenum}
        \item stronger partial charges on H in H bonds. If non-polar, the bond
            is neither acidic nor basic
        \item bond strength, increases as you move to the left-bottom.
            \ce{HBr} is very strong
        \item The greater the stability of the conjugate base \ce{A^-}, the
            stronger the acid.
        \end{compactenum}
    \item[Binary acids] bond strength decreases and acidity and size increase
        down a group.
    \item[Oxyacids] \ce{O-H} bonds present, but the compound is an acid
    \item[OH bond: acid or base] as the electro-negativity of Y in \ce{Y-O-H}
        increases, so does the acidity. Electron density is drawn to \ce{Y}
        so the \ce{O-H} bond becomes weaker and more polar. Also, the stability
        of the conjugate base (\ce{YO^-})increases with the electro-negativity
        of \ce{Y}.
    \item[Acid strength increases] as additional electronegative atoms bond to
        the central atom Y. \[
        \ce{HClO4}
        >
        \ce{HClO3}
        >
        \ce{HClO2}
        >
        \ce{HClO}
        \]
    \item[Carboxylic Acids]
        Contain the carboxyl group \ce{COOH} \chemfig{C ([:90]=O^{-})-O-H}.
        Largest group of organic acids. The conjugate base is stabilized
        by resonance between the two oxygens, spreads negative charge. Also the
        non \ce{O-H} oxygen draws electron density from the broken bond,
        increases polarity.
    \item[Carboxylic acid examples] \ce{CH3COOH}, Benzoic acid
        (benzene and carboxyl), Formic acid \ce{HCOOH}.
\end{compactdesc}
\end{multicols}
\end{mdframed}



\begin{mdframed}
\begin{multicols}{2}
\subsection{Lewis Acids and Bases}
\begin{compactdesc}
    \item[Lewis acids and bases] is a general definition of acids and bases.
    \item[Lewis Acid] electron-pair acceptor
    \item[Lewis Base] electron-pair donor
    \item[Water] is not required. A wider variety of reactions may be treated,
        including acid-base reactions (no proton transfer).
    \item[Strength of electrostatic] interactions
    \item[The interaction of lone pairs] on one molecule with vacant orbitals
        on another is one of the most important concepts in chemistry.
\end{compactdesc}
\end{multicols}
\end{mdframed}







