\section{Chapter 7. Periodic Properties of the Elements}

\secttoc

{\footnotesize
\begin{multicols}{3}
\begin{compactenum}
    \item Explain effective nuclear charge $Z_{eff}$
    \item Trends in atomic/ionic radii, ionization energy, electron affinity
    \item Radius change on loss or gain of $e^-$
    \item Electron configuration of ions
    \item Change in ionization energy as $e^-$ are added or removed, especially
        core electrons
    \item Irregulairites in periodic table, electron affinity and configuration
    \item Differences in metals and nonmetals (basicity of metal oxides,
        acidity of nonmetal oxides)
    \item Correlate atomic properties with electorn configuration
    \item Balanced equations for reactions of groups 1A and 2A with water,
        oxygen, hydrogen and halogens
    \item Unique properties of hydrogen
    \item Atomic properties of groups 6A 7A and 8A with chemical reactivity
        and physical properties
\end{compactenum}
\end{multicols}
}



\begin{mdframed}
\subsection{Periodic Properties of the Elements}
\begin{multicols}{2}
\begin{compactdesc}
\item[Effective nuclear charge] $Z_{eff}$ roughly equal to $Z - S =$ number of
    protons - number of nonvalence $e^-$. The 1s orbital acts as a
    ``lampshade'' to the 2s, 2p orbitals, and so on\dots \textbf{Trend:} increases to
    the right.
\item[Ionization energy] It becomes harder to remove successive $e^-$.
    I = \ce{X (g) -> X^+ (g) + e^-}.
    I$_2$ = \ce{X^+ (g) -> X^{2+} (g) + e^-}.
    \textbf{Trend:} increases to the top-right.
\item[Electron Affinity] Energy for reaction \ce{X (g) + e^- -> X^- (g)}.
    If positive, ions are unstable. \textbf{Trend:} increases to the top-right.
    \ce{F} and \ce{Cl} have the highest.
\item[Atomic radius] Cations are smaller, anions are larger. \textbf{Trend:} increases
    to the bottom-left.
\item[Isoelectric series radius] same charge ions? Size decreases as atomic
    number increases: $\ce{O^{2-}} > \ce{Br^-} > \ce{F^-} > \ce{Al^{3+}}$
\item[$e^-$ configuration of ions] Add or remove $e^-$ from the highest n,
    then l.
\end{compactdesc}
\end{multicols}
\end{mdframed}




\begin{mdframed}
\subsection{Metals}
\begin{multicols}{2}

\begin{compactdesc}
\item[Trend:] metallic character increases to the bottom-left.
\item[Shiny] silver
\item[Malleable and ductile] can be hammered into sheets and stretched into
    wires
\item[Compounds] usually ionic
\item[Solid conductors] of both heat and electricity
\item[Only one liquid] at room temperature: \ce{Hg}
\item[Form cations] in aqueous solution, tend to make basic solutions.
\item[Low first] ionization energy $I_1$
\item[Common reactions]
 \[\ce{oxide + H2O -> hydroxide}   \]
 \[\ce{oxide + acid -> salt + H2O} \]
\end{compactdesc}

\subsection{Alkali metals}
\begin{compactdesc}
\item[Soft solids] naturally present only in compounds
\item[Good conductors]
\item[Low] densities, melting points
\item[Very] reactive, colorful flames when burned
\item[Common reactions]
    \[\ce{2M (s) + H2 (g) -> 2MH (s) }\]
    \[\ce{2M (s) + S (s) -> M2S (s) }\]
    Vigorous reaction in water:
    \[\ce{2M(s) + 2H2O(l) -> 2MOH(aq) + H2(g)}\]
\item[Special with Oxygen] can coerce oxygen to form peroxide: \ce{2Na + 2O -> Na2O2}
    or super peroxide: \ce{K + O2 -> KO2}
\end{compactdesc}

\subsection{Alkaline metals}
\begin{compactdesc}
\item[Solid] release colorful flame when burned
\item[Compared to Alkali metals] harder, denser, less reactive
\item[Water reactions] Be inert; Mg slowly, faster with steam; all others react
    slowly with water
\item[Tend to lose] their two outer $s$ $e^-$.
\end{compactdesc}

\end{multicols}
\end{mdframed}


Metalloids in between.


\begin{mdframed}
\subsection{Nonmetals}
\begin{multicols}{2}
\begin{compactdesc}
\item[Many colors] but no luster
\item[Usually brittle]
\item[Poor conductors] heat and electricity
\item[Nonmetal oxides] molecules. Form acidic solutions.
\item[High first] ionization energy $I_1$
\item[Common reactions]
    \[\ce{oxide + H2O -> acid} \]
    \[\ce{oxide + base -> salt + H2O} \]
\end{compactdesc}

\subsection{Noble Gases}
\begin{compactdesc}
\item[Monatomic] almost no reactions
\item[Filled] s and p orbitals
\item[Possible compounds] exist in rare, laboratory conditions
    \ce{XeF_{2/4/6}}, \ce{KrF2}, \ce{HArF}
\end{compactdesc}



\subsection{Hydrogen}
\begin{compactdesc}
\item[Resembles a] nonmetal more than an Alkali.
\item[Preserves] $e^-$, tends to covalent bonds
\item[Most stable] \ce{H2(g)}
\item[Proton] \ce{H+} present in water
\end{compactdesc}


\subsection{Oxygen group}
\begin{compactdesc}
\item[Atypical nonmetals]
\item[Oxygen stable as] \ce{O2}
\item[Peroxide] \ce{O2^-}
\item[Superperoxide] \ce{O2^{2-}}
\item[Sulfur stable as] \ce{S8}
\item[Stability of water-like] $\ce{H2O} > \ce{H2S} > \ce{H2Se} > \ce{H2Te}$
\item[Possible reaction] air pollutant:
    \[\ce{S(s) + O2(g) -> SO2 (g)}\]
\end{compactdesc}



\subsection{Halogens}
\begin{compactdesc}
\item[Typical] nonmetals
\item[Very] soluble, negative $e^-$ affinity
\item[Fluoride] is extremely reactive
\item[Diatomic] molecules formed such as \ce{I2}, \ce{Cl2}, except F
\item[Trend:] melting, boiling points increase as elements get heavier
\item[Common reactions]
    \[\ce{H2(g) + X2 -> 2HX (g)}\]
    \[\ce{Cl2(g) + H2O(l) -> HCl(aq) + HOCl(aq)}\]
\end{compactdesc}


\end{multicols}
\end{mdframed}




