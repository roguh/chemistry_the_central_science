\section{Chapter 2. Atoms, Molecules, Ions}

\secttoc

{\footnotesize
\begin{multicols}{3}
\begin{compactenum}
    \item Dalton's atomic theory
    \item Key experiments that led to discovery of atom/nucleus/electrons
        (Cathode ray, oil drop, $\alpha$ scattering)
    \item Electrical charge and relative masses of $e^-, p^+, n$
    \item Subatomic composition of isotopes
    \item Atomic weight knowing natural abundances
    \item Periodic table properties, metals/nonmetals
    \item Molecular and ionic substances, in terms of their composition
    \item Empirical/molecular formulas
    \item Compositions expressed as molecular and structural formulas
    \item Ions and the gain/loss of $e^-$, predict common charges
    \item Write empirical formulas of ionic compounds given charges of component ions
    \item Name ionic compounds
    \item Name binary inorganic compounds and acids
    \item Name alkanes and alcohols
\end{compactenum}
\end{multicols}
}
\begin{mdframed}
\begin{multicols}{3}
\subsection{Basics}
\begin{compactdesc}
\item[Empirical formula] extra information is needed to find chemical formula
\item[Mass of electron] $9.10 \cdot 10^{-28}g$
\item[Charge of electron] $1.602 \cdot 10^{-19}C$
\item[Rays] $\beta$ are negative, $\alpha$ are positive, and $\gamma$ are
    neutral.
\item[Cations] positive. meow
\item[Anions] negative.
\end{compactdesc}
\end{multicols}
\end{mdframed}







\begin{mdframed}
\subsection{Dalton's Theory and its Development}
\begin{multicols}{2}
\begin{compactdesc}
\item[Laws:] elements made of atoms, atoms are unique/exclusive to elements,
    atoms cannot be changed by chemistry, compounds have the same relative
    number and kinds of atoms, and the next law:
\item[Law of multiple proportions] elements $A$ and $B$ form two compounds $C_1$
    and $C_2$. This proportion will be a natural number: $\frac{C_1.A}{C_2.A}$
    where $C_1$ is larger.
\item[Cathode rays] cathode rays are electrically charged and negative.
    Determined by firing such a ray and manipulating its positive end using
    magnetic and electric fields
\item[Oil drop experiment] Rate of oil drops affects their rate of descent.
    Used to find ratio of electron's mass and charge.
\item[Rutherford's $\alpha$ scattering] most $\alpha$ particles went through
    a gold foil with little deflection, though some were greatly repelled.
\end{compactdesc}
\end{multicols}
\end{mdframed}



\begin{mdframed}
\subsection{Naming}
\begin{multicols}{2}
Chromate is \ce{CrO4^{2-}}
and Dichromate is \ce{Cr2O7^{2-}}. Permanganate is \ce{MnO4^-}.
Also, the only nonmetal cation encountered is Ammonium \ce{NH4+}.

    \subsubsection{Cations}
    \begin{compactdesc}
    \item[Metal single charge] name ion
    \item[Metal ambiguous charge] name (charge in Roman numerals) ion
    \end{compactdesc}

    \subsubsection{Anions}
    \begin{compactdesc}
    \item[Single] name-ide
    \item[Oxyanions normal sequence] \ce{CO3^{2-}}, \ce{NO3^-}, \ce{ClO3^-};
        \ce{PO4^{3-}}, \ce{SO4^{2-}}
    \item[Oxyanions +1 O count] per-name-ate
    \item[Oxyanions normal O count] name-ate
    \item[Oxyanions -1 O count] name-ite
    \item[Oxyanions -2 O count] hypo-name-ite
    \item[Oxyanions with hydrogen] prepend ``hydrogen'' or ``dihydrogen'' to
        oxyanion name
    \end{compactdesc}

    \subsubsection{Binary compound}
    \begin{compactdesc}
    \item[Greek prefix] to indicate atomic number, leave first alone if AN = 1
    \item[Leftmost first] unless oxygen and a halogen, except \ce{F}.
    \item[Bottommost first]
    \item[Suffix -ide] on second element
    \end{compactdesc}

    \subsubsection{Acids} make \ce{H^+} in water.
    \begin{compactdesc}
    \item[name-ide is converted to] hydro-name-ic acid
    \item[name-ate] is converted to name-ic acid
    \item[name-ite] is converted to name-ous acid
    \end{compactdesc}
\end{multicols}
\end{mdframed}





\begin{mdframed}
\subsection{Alkanes and Alcohols}
\begin{multicols}{2}
\begin{compactdesc}
\item[General formula] \ce{C_nH_{2n + 2}}
\item[Naming sequence] methanol, ethanol, propanol, tetranol, pentanol,
    greek-prefix-nol.
\item[Alcohols] one of the \ce{H} in the alkane replaced by a hydroxide
    \ce{OH}.
\item[Enantiomers] different structure (even if rotated) due to location of the
    hydroxide group. 1-propanol has \ce{OH} on edge, 2-propanol has it in
    center
\end{compactdesc}
\end{multicols}
\end{mdframed}






