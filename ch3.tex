\section{Chapter 3. Chemical Reactions and Reaction Stoichiometry}

\secttoc

{\footnotesize
\begin{multicols}{3}
\begin{compactenum}
    \item Balance chemical equations
    \item Combination, cdecomposition, combustion reactions
    \item Formula weights
    \item Grams and moles
    \item Avogadro's number
    \item Empirical and molecular formulas of a compound from percentage composition and MW
    \item ID limiting reactants and calculate amounts consumed/formed
    \item Percent yield of a reaction
\end{compactenum}
\end{multicols}
}

\begin{mdframed}
\begin{multicols}{2}
\subsection{Chemical Equations and Reactions}
\begin{compactdesc}
\item[Unbalanced Formula] \ce{CH4 + O2 -> CO2 + H2O}
\item[Balanced Formula] \ce{CH4 + 2O2 -> CO2 + 2H2O}
\item[State indicators] (s), (l), (g), (aq) indicates dissolution in water,
    others for different solvents
\item[Formula weight]
    $\sum_\text{Element} \#\text{E} \cdot \text{E atomic weight}$
\item[Elemental Composition] $\frac{\text{\# element} \cdot \text{weight of}}
                                   {\text{formula weight}}$
\item[] empirical formula weight divided by the molecular weight is always
    a natural number
\item[Combination reaction] \ce{A + B -> C}
\item[Decomposition reaction] \ce{C -> A + B}
\item[Limiting Reactants] test each reaction by moles until an obvious limit is
    found
\item[One AMU] is one $\frac{g}{mol}$.
\end{compactdesc}
\end{multicols}
\end{mdframed}






\begin{mdframed}
\begin{multicols}{2}
\subsection{Stoichiometry}
\begin{compactdesc}
\item[One mole] is equal to $6.022 \cdot 10^{23}$ units.
\item[Stoichiometric Equivalence] \ce{A + 2B -> 3C + 4D} one mole of A
    yields 3 moles of C, one mole of D must have consumed half a mole of B.
\item[Combustion analysis] \ce{\dots C \dots H -> nCO2 + mH2O}
    \%C = $\frac{\text{C}mol}{X mol}$
    \%H = $\frac{\text{H}mol}{X mol}$.
    Try different denominators X until all are natural numbers.
    Use algebra if another element involved.
\item[Percentage yield] actual yield divided by theoretical yield.
\end{compactdesc}
\end{multicols}
\end{mdframed}




