\section{Chapter 5. Thermochemistry}

\secttoc

{\footnotesize
\begin{multicols}{3}
\begin{compactenum}
    \item Interconvert energy units
    \item System vs surroundings in thermodynamics
    \item Internal energy from heat and work. Sign conventions
    \item Concept of a state function, examples
    \item $\Delta H$ from $\delta E, P \delta V$
    \item relate $q_p$ to $\Delta H$. Signs indicate exo/endothermic
    \item Relate $\Delta H$ at constant pressure and the amount
        of substance involved
    \item Heat transfered using temperature measurements and heat capcities/specific heats (calorimetry)
    \item Hess' law to determine enthalpy changes
    \item Standard enthalpies of formation to calculate $\Delta H^{o}$
\end{compactenum}
\end{multicols}
}


\begin{mdframed}\begin{multicols}{2}\subsection{Energy}
\begin{compactdesc}
    \item[Kinetic energy] $\frac{mv^2}{2}$
    \item[Electrostatic energy] between two points, charges Q, distance d,
        $k=8.99E9 Jm/C^2$; $\frac{kQ_1Q_2}{d}$
    \item[1 calorie] 4.184 Joules
    \item[System] area singled out for study
    \item[Surroundings] all else
    \item[Work] $F \times d$
    \item[First Law of Thermodynamics] energy cannot be created or destroyed
\end{compactdesc}\end{multicols}\end{mdframed}


\begin{mdframed}\begin{multicols}{2}\subsection{Enthalpy}
\begin{compactdesc}
    \item[Internal energy] E is the sum of kinetic and potential energies.
    \item[Change] $\Delta E = E_\text{final} - E_\text{initial}$
    \item[Change and Work] $\Delta E = q + w$ heat + work.
    \item[Positive $q$] endothermic, \textbf{negative $q$} exothermic.
    \item[Positive $\Delta E$] means system has gained energy, has received: it's endothermic.
    \item[Negative $\Delta E$] means system has lost energy, has given: it's exothermic.
    \item[State function] result depends only on present state
    \item[Enthalpy] $H = E + PV$ where P is pressure and V is volume. All three
        terms are state functions.
    \item[Change in Enthalpy] $\Delta H = H_\text{products} -
        H_\text{reactants}$ $= q + w - w = q$ equal to heat at constant
        pressure. \textbf{Extensive} property (depends on amount). Depends on
        states of substances. Units: $kJ$
    \item[Example]
        \[  \ce{2H2(g) + O2 (g) -> 2H2O (g)}, \Delta H = -483.6 kJ
        \]
\end{compactdesc}

\subsection{Calorimetry}
Specific heat of substance mass(units $\frac{J}{molK}$), range and change in
temperature can be used to find the heat released or absorbed during a change.
\[Q = C_sm\Delta T\]
\end{multicols}\end{mdframed}



\begin{mdframed}\begin{multicols}{2}\subsection{Hess's law}
\begin{compactdesc}
\item[Composite reaction] can be split into other, sometimes simpler, reactions
    which add to form it
\item[Hess's law] indicates the $\Delta H$ for a composite reaction is the sum
    of the $\Delta H$ for each component reaction.
\end{compactdesc}

\subsection{Enthalpies of Formation}
\begin{compactdesc}
    \item[Enthalpy of formation] $\Delta H_f^\circ$ is the enthalpy required
        to \textbf{form} a substance in standard conditions, $25^\circ$C and 1
        atm. Units: $kJ/mol$.
    \item[For stablest elementals] at standard conditions, such as
        C(graphite), \ce{H2}, \ce{O2}, $\Delta H_f^\circ = 0$.
    \item[Example] for diamond \ce{C(s)} $\Delta H_f^\circ = 1.88 kJ/mol$.
        For water vapor $\Delta H_f^\circ = -241.8 kJ/mol$
\end{compactdesc}
\end{multicols}\end{mdframed}


\begin{mdframed}
\subsection{Foods and Fuels}
\begin{multicols}{2}
\begin{tabular}{ll}
Proteins& 17
    \\
Fats& 38
    \\
Carbohydrates& 17
    \\
\end{tabular}

\begin{tabular}{ll}
Wood, pine& 18
    \\
Charcoal& 34
    \\
Texas Crude Oil& 45
    \\
Hydrogen& 142
    \\
\end{tabular}
\end{multicols}
Non-renewable fuels are fucked, renewable fuels not used enough.
\end{mdframed}



