\section{Chapter 15. Chemical equilibrium}


\secttoc

{\footnotesize
\begin{multicols}{3}
\begin{compactenum}
    \item What is meant by equilibrium, relate to rates
    \item Write equilibrium-constant expression for any reaction
    \item Inter-convert $K_c$ and $K_p$
    \item Magnitude of equilibrium constant and relative amounts of reactants
        and products
    \item Manipulate constant to reflect changes in chemical equation
    \item Heterogeneous reaction equilibrium constant (EC)
    \item Calculate EC from concentration measurements
    \item Predict direction of a reaction given EC and conc.s
    \item Calculate concentrations given EC and one equi. conc.
    \item Calculate equi. conc.s given EC and starting conc.s
    \item Le Ch\^{a}telier's principle to predict how conc.s, volume or temperature
        of system at equilibrium affects the equilibrium position
\end{compactenum}
\end{multicols}
}

\begin{mdframed}
\begin{multicols}{2}
\subsection{Rock bottom}
\begin{compactdesc}
    \item[Dynamic equilibrium] two opposing events have the same rate, no net
        change occurs.
    \item[Vapor pressure] rate of molecules leaving the liquid near the surface
        is equal to the rate of their return to liquid state
    \item[Chemical equilibrium] opposing reactions with the same rate.
    \item[Example] \ce{N2O4 (g) (colorless) <=> 2NO2 (g) (brown)}. The
        equilibrium constant is \[
            k = \frac{k_f}{k_r} = \frac{[\ce{NO2}]^2}{[\ce{N2O4}]}
        \]
    \item[At equilibrium] concentrations do not change and products/reactants
        cannot escape system.
    \item[For equilibrium] to occur, both reactions must be able to occur.
\end{compactdesc}
\end{multicols}
\end{mdframed}


\begin{mdframed}
\begin{multicols}{2}
\subsection{Equilibrium constant}
\begin{compactdesc}
    \item[Unique to an equilibrium] even if initial concentrations change
    \item[law of mass action] consider the general equilibrium
        equation \ce{aA + bB <=> dD + eE}. The equilibrium expression is
        \[
            K_c = \frac { [\ce{D}]^d [\ce{E}]^e} {[\ce{A}]^a [\ce{B}]^b }
        \]
    \item[No units] $k_c$ is unitless because another method of measuring it
        ends up unitless.
    \item[Activity] of a substance in an ideal mixture is the ratio of $C$ or
        $P$ to a reference like $1M$ or $1atm$. $A = C M / 1 M$ thus, no units
    \item[Solids and Liquids] activity is unity.
    \item[Depends on order] of the equilibrium reaction, even though they
        are naturally order-less. Thus, in a way $K_c = \frac{1}{K_c}$.
    \item[Can use partial pressure] instead of concentrations for gaseous
        reactions.
    \item[Converting] $K_p = K_c (RT)^{\Delta n}$ where $\Delta n = $ moles
        of gaseous product -- moles of gaseous reactant.
\end{compactdesc}
\end{multicols}
\end{mdframed}


\begin{mdframed}
\begin{multicols}{2}
\subsection{Direction and Summation of Equations}
\begin{compactdesc}
    \item[Reaction written backwards] $K_{rev} = \frac{1}{K}$
    \item[Reaction multiplied] by a constant $n$, $K_{n} = K^n$.
    \item[Reactions added] simply multiply all $K$s involved.
        Adding the two equations:
        \[ \ce{aA + bB <=> dD + cC} \] with $K_0$ and
        \[ \ce{cC + fF <=> gG + aA} \] with $K_1$ will give
        \[ \ce{bB + fF <=> dD + gG} \] with $K_0 K_1$.
\end{compactdesc}
\end{multicols}
\end{mdframed}


\begin{mdframed}
\begin{multicols}{2}
\subsection{Heterogeneous Equilibrium}
\begin{compactdesc}
    \item[homogeneous] all involved substances in same phase
    \item[heterogeneous] involved substances in different phases
    \item[liquid or solid] substances have an activity equal to unity.
        Ratio of their mass to volume is constant.
\end{compactdesc}
\end{multicols}
\end{mdframed}


\begin{mdframed}
\begin{multicols}{2}
\subsection{Finding K from initial and equilibrium concentrations}
\begin{compactdesc}
    \item[If initial and final known] then $\Delta conc$ is known.
    \item[If initial and another $\Delta conc$ known] use coefficients in balanced
        reaction to relate change in the known with the current unknown.
    \item[If initial and change known] just add!
    \item[Use an ICE table] Initial, Change in, and Equilibrium
        concentrations.
    \item[Example]
        Key steps: finding change in $conc$ knowing only $\Delta conc_{\ce{HI}} $.
        \[
            \Big( 1.87\cdot 10^{-3} \Big)
            \Big( \frac{ 1 mol \ce{H2} } { 2 mol \ce{HI} } \Big)
            = 0.935\cdot 10^{-3}
        \]
\end{compactdesc}
\end{multicols}
\begin{table}[H]
    \begin{tabular}{lccccc}
        & \ce{H2 (g)} & + & \ce{I2 (g)} & \ce{<=>} & \ce{2HI (g)} \\
        Initial & 1.00$\cdot 10^{-3}$ & & 2.00$\cdot 10^{-3}$ & & $0.00$ \\
        Change  & \textbf{-0.935$\cdot 10^{-3}$} & & \textbf{-0.935$\cdot 10^{-3}$} & & \textbf{+1.87$\cdot 10^{-3}$} \\
        Final   & \textbf{0.065$\cdot 10^{-3}$} & & \textbf{1.065$\cdot 10^{-3}$} & & 1.87$\cdot 10^{-3}$ \\
    \end{tabular}
    \centering
    \end{table}
\end{mdframed}


\begin{mdframed}
\begin{multicols}{2}
\subsection{Applications of K}
\begin{compactdesc}
    \item[Predicting the Direction of Reaction] If $Q < K$ then more products
        will be needed, if $Q > K$ then more reactants needed.
    \item[Reaction quotient] Calculated like $K$, but for concentrations or
        partial pressures at any point in the reaction.
    \item[Calculating equilibrium concentrations] just solve for unknowns using
        known equilibrium concentrations.
\end{compactdesc}
\end{multicols}
\end{mdframed}


\begin{mdframed}
\subsection{Le Ch\^{a}telier's Principle}
\begin{multicols}{2}
\begin{compactdesc}
    \item[Le Ch\^{a}telier's Principle] if equilibrium is disturbed in any way,
        the system will shift its position to counteract. ``le-SHOT-lee-ay.''
    \item[Lies to the right] lots of ``product''
    \item[Lies to the left] lots of ``reactant''
    \item[Concentration of reactants increased] increases concentration of products
    \item[Pressure change] by changing volume, if increased in gaseous equilibrium the system
        will want to minimize the number of moles of gas. Constant temperature.
    \item[Volume change] same as pressure change
    \item[Temperature change] If reaction is endothermic, increasing T
        increases $K$. If reaction is exothermic, increasing T decreases $K$.
    \item[Effects of catalysts] activation energies are lowered,
        but $K$ is not changed. Equilibrium will be reached faster.
\end{compactdesc}
\end{multicols}
\end{mdframed}



