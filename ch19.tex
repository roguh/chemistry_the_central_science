\setcounter{section}{18}
\section{Chapter 19. Chemical Thermodynamics}
\secttoc

{\footnotesize \begin{multicols}{3}\begin{compactenum}
\item Spontaneous, reversible, irreversible and isothermal processes
\item Entropy and the Second law
\item Entropy and micro-states
\item Possible molecular motions
\item Predict sign of $\Delta$S for physical and chemical
\item The Third law
\item Standard entropy changes using standard molar entropies
\item Entropy changes for an isothermal process
\item Gibbs free energy from enthalpy change
\item Entropy change at a temperature
\item Free-energy changes to predict if spontaneous
\item Effect of temperature on spontaneity given $\Delta H$ and $\Delta S$
\item $\Delta G$ under nonstandard conditions
\item $\Delta G^\circ$ and equilibrium constant
\end{compactenum}\end{multicols}}


\begin{mdframed}
\begin{multicols}{2}
\subsection{Spontaneous Processes}
\begin{compactdesc}
    \item[the first law] helps us keep track of heat changes, but does not
        tell us whether a proces is favored because of anything we did to the
        system. $\Delta H = q + w$.
    \item[thermodynamics] all about direction and extent of a reaction, not
        about rate.
    \item[spontaneous] inherently directional process. Related to thermodynamic
        path from start to end states.
    \item[example] ice melts to water spontaneously at a high enough temperature.
    \item[reversible process] system and surroundings can return to original state by
        an exact reversal. Truly, they must occur with infinitesimally small
        units of heat occurring infinitesimally slowly.
    \item[irreversible process] system cannot return to original state without
        permanent change in surroundings
    \item[isothermal] process occurs at same temperature
\end{compactdesc}
\end{multicols}
\end{mdframed}






\begin{mdframed}
\begin{multicols}{2}
\subsection{Entropy and the Second Law}
\begin{compactdesc}
    \item[entropy] denoted $S$, at constant temperature $\Delta S = q_{rev}/T$
        where $q_{rev}$ is the heat the process would absorb if it was
        reversible.
        Units $J/K$. State function.
    \item[for any process] $\Delta S_{univ} = \Delta S_{sys} + \Delta S_{sur}$
    \item[for phase changes] $\Delta q_{rev} = \Delta H_{fusion}$.
    \item[isothermic gas expansion] $w_{rev} = -nRT\ln \frac{V_2}{V_1}$
    \item[the second law] $\Delta S_{univ} = 0$ for reversible processes and
        $\Delta S_{univ} > 0$ for irreversible processes.
\end{compactdesc}
\end{multicols}
\end{mdframed}






\begin{mdframed}
\begin{multicols}{2}
\subsection{Molecular Interpretation of Entropy and the Third Law}
\begin{compactdesc}
    \item[microstate] combination of motions and locations of atoms.
    \item[Boltzmann's] $S=k \ln W$ where k is B's constant
    \item[entropy] randomness or disorder of a system. Related to
        number of microstates.
    \item[translational motion] entire molecule moves. Kinetic energy.
    \item[vibation motion] periodic molecular motion. ``accordion'' bonds
    \item[rotational motion] molecule spins like a top
    \item[entropy increases] with increase in volume, temperature, motion of
        molecules, increase of motions and locations of molecules.
    \item[entropy increases] solid dissolves, phase change
        $(s) \to (l) \to (g)$ or increase in molecules.
    \item[the third law] entropy of a pure crystalline solid at 0K is zero. W = 1.
        No motion, absolute zero, no entropy.
\end{compactdesc}
\end{multicols}
\end{mdframed}






\begin{mdframed}
\begin{multicols}{2}
\subsection{Entropy Changes in Chemical Reactions}
\begin{compactdesc}
    \item[standard molar entropy] denoted $S^\circ$ is entropy of a mole of
        substance at standard conditions.
    \item[general observations] Not the same as enthalpies of
        formation! Elements are reference temperature are not zero.
        Increase with increasing molar mass, or increasing number of atoms in
        formula. Gases greater than liquids or solids.
    \item[tabulated $\Delta S^\circ$] can be used to calculate entropy change of
        any reaction.
        $\Delta S^\circ = \sum_{products} nS^\circ - \sum_{reactants} mS^\circ$
    \item[entropy change in surroundings] for isothermal process is
        $\Delta S = -\Delta H/T = k \ln \frac {W_2} {W_1} $.
\end{compactdesc}
\end{multicols}
\end{mdframed}






\begin{mdframed}
\begin{multicols}{2}
\subsection{Gibbs Free Energy}
\begin{compactdesc}
    \item[Gibbs free energy] thermodynamic state function combining two state
        functions $G = H - TS$. For isothermal processes: $\Delta G = \Delta H
        - T \Delta S$.
    \item[Gibbs at constant] temperature and pressure indicates spontaneity.
        Negative $\Delta G$ means spontaneous, positive means nonspontaneous
        but reverse process is spontaneous.
    \item[Equilibrium] $\Delta G = 0$. A spontaneous process.
    \item[maximum level of work] that can be performed by system indicated by
        the free energy. $\Delta G = -w_{max}$.
    \item[standard free energies of formation] $\Delta G^\circ_f$ defined
        just like standard enthalpies of formation. Defined zero for a pure
        element in standard state (convention, we only care about changes).
        Can be used to find standard free-energy change $\Delta G^\circ$.
        \[
            \Delta G^\circ = \sum_{products}  n\Delta G^\circ_f
                           - \sum_{reactants} m\Delta G^\circ_f
        \]

\end{compactdesc}
\end{multicols}
\end{mdframed}






\begin{mdframed}
\begin{multicols}{2}
\subsection{Free Energy, Temperature, and Equilibrium Constant}
\begin{compactdesc}
    \item[temperature doesn't affect] $\Delta H$ and $\Delta S$ of a process
        very much. $\Delta G$ is governed by temperature.
    \item[entropy term $-T\Delta S$] has greater effect on temperature
        dependence. Process with both $\Delta H > 0$ and $\Delta S > 0$ can
        be nonspontaneous at low temperatures but spontanoues at high
        temperatures. Example: ice.
    \item[at equilibrium] $\Delta G = 0, Q = K$ thus this equation:
        $\Delta G = \Delta G^\circ + RT\ln Q$ turns into equation directly
        dependant on temperature and standard free-energy change
        \[
            \Delta G^\circ = -RT\ln K
        \]
        \[
            K = e^{- \Delta G^\circ / RT}
        \]
        \[
            \Delta G = \Delta G^\circ + RT\ln K
        \]
\end{compactdesc}
\end{multicols}
\end{mdframed}






